%%%%%%%%%%%%%%%%%%%%%%%%%%%%%%%%%%%%%%%%%%%%%%%%%%%%%%%%%%%%%%%%%%%%%%%%%%%%%%

\begin{frame}
  \frametitle{Présentation}

  \begin{itemize}
  \item \LaTeX{} est un outil de composition
    
    \begin{enumerate}
    \item on prépare les sources avec un éditeur (Texmaker par exemple)
    \item on compile pour produire un fichier PDF
    \end{enumerate}

    \pause
    
  \item Beamer est un paquet \LaTeX{} pour faire des présentations
    
    \begin{enumerate}
    \item on crée la structure (plan) du cours
    \item on ajoute ensuite les diapos
    \end{enumerate}

    \pause

  \item \LaTeX{} et Beamer séparent la définition du fond et de la forme
    \begin{center}
      \begin{alertenv}
        Il faut se concentrer sur la rédaction du fond
      \end{alertenv}
    \end{center}

    \pause

  \item Préparer un cours avec \LaTeX{} et Beamer est simple

    \begin{itemize}
    \item il ne faut (d'abord) pas s'occuper de la forme
    \item il faut avoir un peu de \alert{discipline}
    \end{itemize}
    \note{On peut mettre des notes visibles juste par l'orateur}
  \end{itemize}
\end{frame}

%%%%%%%%%%%%%%%%%%%%%%%%%%%%%%%%%%%%%%%%%%%%%%%%%%%%%%%%%%%%%%%%%%%%%%%%%%%%%%

\begin{frame}
  \frametitle{Découpage en plusieurs fichiers}

  \begin{itemize}
  \item Une présentation peut-être composée grâce à plusieurs fichiers sources
  \item Le fichier principal est celui qui doit être compilé

    \begin{itemize}
    \item Il fait référence à d'autres fichiers qu'il inclut pendant la compilation
    \item Il est le seul à avoir besoin des entêtes de configuration
    \end{itemize}

  \item Ça a plein d'avantages

    \begin{itemize}
    \item On évite d'avoir un seul gros fichier
    \item On peut réutiliser facilement des parties de présentation 
    \end{itemize}

    \pause

  \item Sous Texmaker, lors de chaque session de travail

    \begin{itemize}
    \item Ouvrir le fichier principal
    \item Le définir comme document principal

      \begin{itemize}
      \item menu «\emph{Options}»,
      \item choix «\emph{Définir le document courant comme document 'maître'}»
      \end{itemize}

    \item Ouvrir le(s) fichier(s) voulu(s) et les compiler normalement
    \end{itemize}

  \end{itemize}
\end{frame}

%%%%%%%%%%%%%%%%%%%%%%%%%%%%%%%%%%%%%%%%%%%%%%%%%%%%%%%%%%%%%%%%%%%%%%%%%%%%%%
