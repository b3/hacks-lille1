%%%%%%%%%%%%%%%%%%%%%%%%%%%%%%%%%%%%%%%%%%%%%%%%%%%%%%%%%%%%%%%%%%%%%%%%%%%%%%

\begin{frame}[fragile]
  \frametitle{Les outils à installer}


  \structure{Un système \TeX{} avec les paquets nécessaires}

  \begin{itemize}
  \item \TeX{} Live \hfill \url{http://www.tug.org/texlive}
  \item Mac\TeX{} \hfill \url{https://www.tug.org/mactex}
  \end{itemize}
  
  \pause

  \structure{Un éditeur facilitant le processus \emph{édition$\rightarrow$compilation$\rightarrow$visualisation}}

  \begin{itemize}
  \item Texmaker \hfill \url{http://www.xm1math.net/texmaker}
    
  \item Emacs avec AUC\TeX{} \hfill \url{https://www.gnu.org/software/emacs}
    
    \hfill \url{https://www.gnu.org/software/auctex}
    
  \end{itemize}
    
  \pause
  
  \structure{Un outil/service en ligne qui empaquete le tout (et un peu plus)}

  \begin{itemize}
  \item \textsc{ShareLaTeX} \hfill \url{https://www.sharelatex.com}
  \item $\backslash{}$BlueLaTeX \hfill \url{http://www.bluelatex.org}
  \end{itemize}
\end{frame}

%%%%%%%%%%%%%%%%%%%%%%%%%%%%%%%%%%%%%%%%%%%%%%%%%%%%%%%%%%%%%%%%%%%%%%%%%%%%%%

\begin{frame}[allowframebreaks]
  \frametitle<presentation>{Pour aller plus loin}
    
  \begin{thebibliography}{10}
    
    \beamertemplateonlinebibitems
    
  \bibitem{WikiBooks}
    \LaTeX{} Wikibooks
    \newblock \url{https://en.wikibooks.org/wiki/LaTeX}
      
  \bibitem{latexproject}
    The \LaTeX{} project
    \newblock \url{https://www.latex-project.org}

    \beamertemplatebookbibitems
    
  \bibitem{LaTeXBook}
    Leslie Lamport
    \newblock \emph{LaTeX: A document preparation system. User's guide and reference manual}
    \newblock Addison-Wesley, 1994
    
    \beamertemplatearticlebibitems

  \bibitem{Un papier}
    Prénom NOM
    \newblock Un article qui parle de LaTeX
    \newblock \emph{Journal of This and That}, 2(1):50--100,

  \end{thebibliography}
\end{frame}
  
%%%%%%%%%%%%%%%%%%%%%%%%%%%%%%%%%%%%%%%%%%%%%%%%%%%%%%%%%%%%%%%%%%%%%%%%%%%%%%