\documentclass[10pt,t]{beamer}

% Paquets LaTeX %%%%%%%%%%%%%%%%%%%%%%%%%%%%%%%%%%%%%%%%%%%%%%%%%%%%%%%%%%%%%%

%% Une gestion correcte du français (en entrée et en sortie)
\usepackage[french]{babel}
\usepackage{type1ec}         % devant fontenc (cf type1ec.sty)
\usepackage[T1]{fontenc}     % devant inputenc (utf8 choisi en fonction de ça)
\usepackage[utf8]{inputenc}
\DeclareUnicodeCharacter{20AC}{\euro} % pour la saisie du caractère euro

%% Des "jolies" polices de caractères
\usepackage{lmodern}            % pour sf et tt
\usepackage{fourier}            % pour rm
\usepackage{bbm}                % pour les mathbbm

%% Plein de symboles
\usepackage{amssymb}            % Les symboles mathématiques de l'AMS
\usepackage{latexsym}           % Quelques symboles manquants dans LaTeX 2e
\usepackage{marvosym}           % Quelques symboles en vrac par Martin Vogel
\usepackage{wasysym}            % Quelques symboles en vrac par Roland Waldi
\usepackage{pifont}             % Les symboles Dingbats
\usepackage{textcomp}           % \textcopyleft
\usepackage[copyright]{ccicons} % Les (c) comme dans Creative Commons
\usepackage[official,right]{eurosym} % L'euro

%% Quelques paquets utiles
\usepackage{array}              % pour faciliter les styles de tableaux
\usepackage{relsize}            % pour faciliter le changement de taille des polices
\usepackage[normalem]{ulem}     % pour avoir des soulignements funky
\usepackage{tikz}               % pour les dessins portables
\usepackage{pgfpages}           % pour les présentations en double-écran

% Paramétrages Beamer %%%%%%%%%%%%%%%%%%%%%%%%%%%%%%%%%%%%%%%%%%%%%%%%%%%%%%%%
\usetheme{lille1}
%\setbeameroption{show notes on second screen} % les notes sur le second écran

% Méta-données du document %%%%%%%%%%%%%%%%%%%%%%%%%%%%%%%%%%%%%%%%%%%%%%%%%%%

\title{Un modèle de cours à Lille~1}
\subtitle{Avec \LaTeX{} et Beamer}
\author{Prénom NOM}
\institute{Université Lille~1 \and IUT «A»}
\date{2015/2016}

%%%%%%%%%%%%%%%%%%%%%%%%%%%%%%%%%%%%%%%%%%%%%%%%%%%%%%%%%%%%%%%%%%%%%%%%%%%%%%

\begin{document}

\begin{frame}[plain,label=titre]
  \titlepage
\end{frame}

\begin{frame}
  \frametitle{Plan}
  \tableofcontents
\end{frame}

%%%%%%%%%%%%%%%%%%%%%%%%%%%%%%%%%%%%%%%%%%%%%%%%%%%%%%%%%%%%%%%%%%%%%%%%%%%%%%

\begin{frame}% [shrink] si on veut forcer à tenir sur une seul frame
  \frametitle{Introduction}

  \structure{Un exemple}

  \begin{itemize}
  \item Cette présentation est produite par \texttt{\textbf{modele-cours-lille1.tex}}
  \item Elle est conçue pour des cours magistraux à Lille~1 pour des gens d'une composante
  \item Son contenu est un exemple pour comprendre comment

    \begin{itemize}
    \item composer avec \LaTeX{}
    \item préparer une présentation avec Beamer
    \end{itemize}

  \end{itemize}

  \pause

  \structure{Un modèle}

  \begin{itemize}
  \item Elle constitue un modèle pour faire d'autres cours :
    
    \begin{enumerate}
    \item créer un nouveau dossier ;
    \item y copier les fichiers 

      \begin{itemize}
      \item \texttt{\textbf{beamerthemelille1.sty}}, qui est le thème Beamer pour Lille~1,
      \item \texttt{\textbf{logo-univ-lille1.pdf}}, qui est le logo de Lille~1 utilisé,
      \end{itemize}

    \item y copier, renommer puis modifier le fichier \texttt{\textbf{modele-cours-lille1.tex}} ;
    \item y compiler le fichier.
    \end{enumerate}
  \end{itemize}

  \pause

  \structure{La seconde partie est une courte référence \LaTeX{}}
\end{frame}

%%%%%%%%%%%%%%%%%%%%%%%%%%%%%%%%%%%%%%%%%%%%%%%%%%%%%%%%%%%%%%%%%%%%%%%%%%%%%%

\section{Faire une présentation en \LaTeX{} avec Beamer}

\subsection{Les bases}
%%%%%%%%%%%%%%%%%%%%%%%%%%%%%%%%%%%%%%%%%%%%%%%%%%%%%%%%%%%%%%%%%%%%%%%%%%%%%%

\begin{frame}
  \frametitle{Présentation}

  \begin{itemize}
  \item \LaTeX{} est un outil de composition
    
    \begin{enumerate}
    \item on prépare les sources avec un éditeur (Texmaker par exemple)
    \item on compile pour produire un fichier PDF
    \end{enumerate}

    \pause
    
  \item Beamer est un paquet \LaTeX{} pour faire des présentations
    
    \begin{enumerate}
    \item on crée la structure (plan) du cours
    \item on ajoute ensuite les diapos
    \end{enumerate}

    \pause

  \item \LaTeX{} et Beamer séparent la définition du fond et de la forme
    \begin{center}
      \begin{alertenv}
        Il faut se concentrer sur la rédaction du fond
      \end{alertenv}
    \end{center}

    \pause

  \item Préparer un cours avec \LaTeX{} et Beamer est simple

    \begin{itemize}
    \item il ne faut (d'abord) pas s'occuper de la forme
    \item il faut avoir un peu de \alert{discipline}
    \end{itemize}
    \note{On peut mettre des notes visibles juste par l'orateur}
  \end{itemize}
\end{frame}

%%%%%%%%%%%%%%%%%%%%%%%%%%%%%%%%%%%%%%%%%%%%%%%%%%%%%%%%%%%%%%%%%%%%%%%%%%%%%%

\begin{frame}
  \frametitle{Découpage en plusieurs fichiers}

  \begin{itemize}
  \item Une présentation peut-être composée grâce à plusieurs fichiers sources
  \item Le fichier principal est celui qui doit être compilé

    \begin{itemize}
    \item Il fait référence à d'autres fichiers qu'il inclut pendant la compilation
    \item Il est le seul à avoir besoin des entêtes de configuration
    \end{itemize}

  \item Ça a plein d'avantages

    \begin{itemize}
    \item On évite d'avoir un seul gros fichier
    \item On peut réutiliser facilement des parties de présentation 
    \end{itemize}

    \pause

  \item Sous Texmaker, lors de chaque session de travail

    \begin{itemize}
    \item Ouvrir le fichier principal
    \item Le définir comme document principal

      \begin{itemize}
      \item menu «\emph{Options}»,
      \item choix «\emph{Définir le document courant comme document 'maître'}»
      \end{itemize}

    \item Ouvrir le(s) fichier(s) voulu(s) et les compiler normalement
    \end{itemize}

  \end{itemize}
\end{frame}

%%%%%%%%%%%%%%%%%%%%%%%%%%%%%%%%%%%%%%%%%%%%%%%%%%%%%%%%%%%%%%%%%%%%%%%%%%%%%%


\subsection{Des conseils de productions}
%%%%%%%%%%%%%%%%%%%%%%%%%%%%%%%%%%%%%%%%%%%%%%%%%%%%%%%%%%%%%%%%%%%%%%%%%%%%%%

\begin{frame}[fragile]
  \frametitle{Conseils d'édition}

  Avoir des source \LaTeX{} propres permet 

  \begin{enumerate}
  \item de se repérer rapidement dans la présentation
  \item de détecter plus facilement les erreurs de compilation
  \end{enumerate}

  \alert{Il faut s'astreindre à des conventions d'écriture des sources}

  \pause

  \begin{itemize}
  \item Toujours décaler le contenu d'un environnement vers la droite

    \begin{itemize}
    \item Objectif : faire apparaître la structure
    \item Même largeur à chaque fois (2 espaces par exemple)
    \item Additioner les décalages à chaque descente en profondeur      
    \end{itemize}

  \item Spécifité des environnements de liste

    \begin{itemize}
    \item Les items de liste doivent être alignées avec le début et la fin de l'environnement
    \item Laisser une ligne vide devant le début d'une liste
    \item Laisser une ligne vide après la fin d'une liste
    \end{itemize}

  \item Éviter la multiplication des lignes vides qui ne servent à rien
  \item Entourer les diapos avec des longues lignes de commentaires (\%)
  \end{itemize}
\end{frame}

%%%%%%%%%%%%%%%%%%%%%%%%%%%%%%%%%%%%%%%%%%%%%%%%%%%%%%%%%%%%%%%%%%%%%%%%%%%%%%


\section{Une petite référence}

\subsection{\LaTeX{}}
%%%%%%%%%%%%%%%%%%%%%%%%%%%%%%%%%%%%%%%%%%%%%%%%%%%%%%%%%%%%%%%%%%%%%%%%%%%%%%

\begin{frame}
  \frametitle{Courte référence \LaTeX{}}
  \framesubtitle{Listes, emphases, liens}

  \structure{Une liste standard}
  
  \begin{itemize}
  \item un point en \emph{emphase}
    \note{Ici il faut dire un bidule sur l'emphase}
  \item un point \textbf{important}
  \item un autre point \underline{souligné}
  \end{itemize} 
  
  \pause  
  
  \structure{Une liste numérotée}
  
  \begin{enumerate}
  \item un point avec un lien vers le \href{http://www.univ-lille1.fr}{site web de l'Université}
  \item un autre point sur le site web : \url{http://www.univ-lille1.fr}
  \item un choix avec de l'€
  \end{enumerate}

  \pause

  \structure{Une liste de description}

  Quelques environnements permettent des structurations 

  \begin{description}
  \item[itemize] définit une liste standard
  \item[enumerate] définit une liste ordonnée
  \item[tabular] permet de faire des tableaux
  \item[displaymath] permet d'afficher des mathématiques
  \item[equation] permet d'afficher une équation et de la numéroter
  \end{description}
\end{frame}

%%%%%%%%%%%%%%%%%%%%%%%%%%%%%%%%%%%%%%%%%%%%%%%%%%%%%%%%%%%%%%%%%%%%%%%%%%%%%%

\begin{frame}[allowframebreaks] % une grande diapo qui sera découpée auto (pas de pause dedans)
  \frametitle{Courte référence \LaTeX{}}
  \framesubtitle{Mathématiques}

  \structure{3 possibilités de saisir du contenu \emph{mathématiques}}

  \begin{enumerate}
  \item En ligne : $ \sum_{i=0}^{i=10} a_{i} $

  \item En affiche :
    \begin{displaymath}
      \sum_{i=0}^{i=10} a_{i}
    \end{displaymath}

  \item En équation pour y faire référence plus tard
    \begin{equation}
      \label{eq:1}
      F(x) = \int_{a}^{b}\sin(x)dx 
    \end{equation}
  \end{enumerate}
  
  \structure{Des fonctionnalités accessibles uniquement en mode mathématique}

  \begin{itemize}
  \item Les lettres grecs et exotiques sont accessibles facilement :

    \begin{itemize}
    \item en minuscules : $ \alpha, \beta, \gamma, \delta, \epsilon, \zeta, \eta, \theta, \iota, \kappa, \lambda, \mu, \nu, \xi, \pi, \rho, \sigma, \tau, \upsilon, \phi, \chi, \psi, \omega $
    \item en capitales : $ \Gamma, \Delta, \Theta, \Lambda, \Xi, \Pi, \Sigma, \Upsilon, \Phi, \Psi, \Omega $
    \end{itemize}

  \item Les notations d'ensemble sont composables : $ \mathbbm{N}, \mathbbm{Z}^{+}, \mathbbm{Q}, \mathbbm{R} $

  \item Beaucoup de symboles mathématiques et autres sont disponibles

    \begin{itemize}
    \item $ \neq, \approx, \simeq, \le, \ge, \iff, \equiv, \in, \infty, \forall, \exists, \subset, \subseteq, \times, \emptyset, \rightarrow, \qed, \prod, \coprod, \nabla, \partial, \prec, \preceq, \cdots $
    \item $ \spadesuit, \heartsuit, \diamondsuit, \clubsuit, \ell, \sharp, \flat, \cdots $
    \item \url{http://detexify.kirelabs.org} permet de les trouver en les dessinant
    \end{itemize}

  \item On a plein de constructions

    \begin{description}
    \item[racine] $ \sqrt[3]{x + y} $
    \item[fractions] $ \frac{x\times 3}{4} $
    \item[relation] $ x \stackrel{i+j}{\longmapsto} y $
    \item[somme] $ \sum_{0}^{\infty} x_{i} $
    \item[produit] $ \prod_{0}^{\infty} x_{i} $
    \item[intégral] $ \int_{a}^{b} \pi_{i} $
    \item[regroupement]
      $ \underline{dessous} $,
      $ \overbrace{dessus}^{i\rightarrow\infty} $,
      sur le côté
      \begin{math} % pareil que $
        f(x) = \left\{
          \begin{array}{l@{~:~}r}
            x = 0 & x \\
            x \neq 0 & \frac{x}{2}
          \end{array}
        \right.
      \end{math}
    \end{description}

  \item On a des points de suite dans tous les sens $ \dots, \cdots, \vdots, \ddots $
  \end{itemize}
\end{frame}

%%%%%%%%%%%%%%%%%%%%%%%%%%%%%%%%%%%%%%%%%%%%%%%%%%%%%%%%%%%%%%%%%%%%%%%%%%%%%%

\begin{frame}[fragile]
  \frametitle{Courte référence \LaTeX{}}
  \framesubtitle{Avancé}

  \structure{Tableaux}

  \begin{itemize}
  \item Une suite de lignes arrangées en colonnes via \verb|tabular| ou \verb|array| (maths)
  \item Il doit y avoir une ligne blanche devant et derrière
    
    \begin{tabular}{|c|r|l|}
      \firsthline
      colonnes & définition         & paramètre de l'environnement via l, c, r  \\ \cline{2-3}
               & traits verticaux   & \verb!|! dans la définition des colonnes  \\ \hline
      lignes   & terminaison        & $\backslash\backslash$ en fin de ligne    \\ \cline{2-3}
               & traits horizontaux & sur toute la largeur via \verb|\hline|    \\ \cline{2-3}
               &                    & sous certaines colonnes via \verb|\cline| \\ \hline
    \end{tabular}
  \end{itemize}

  \structure{Références aux équations}

  \begin{itemize}
  \item on la nomme avec \verb|\label{}|
  \item on y fait référence avec \verb|\ref{}|
  \item par exemple l'équation \ref{eq:1} utilise une intégrale
  \end{itemize}
\end{frame}

%%%%%%%%%%%%%%%%%%%%%%%%%%%%%%%%%%%%%%%%%%%%%%%%%%%%%%%%%%%%%%%%%%%%%%%%%%%%%%

\begin{frame}[fragile]
  \frametitle{Courte référence \LaTeX{}}
  \framesubtitle{Avancé}

  \structure{Caractères spéciaux}

  \begin{center}
    \begin{tabular}{rl}
      \emph{caractère} & \emph{commande} \\ \hline
      \# & \verb|\#| \\
      \$ & \verb|\$| \\
      \% & \verb|\%| \\
      \& & \verb|\&| \\
      \textasciitilde & \verb|\textasciitilde| \\
      \_ & \verb|\_| \\
      \textasciicircum & \verb|\textasciicircum| \\
      $\backslash$ & \verb|$\backslash$| \\
      \{ & \verb|\{| \\
      \} & \verb|\}| \\
    \end{tabular}
  \end{center}
\end{frame}

%%%%%%%%%%%%%%%%%%%%%%%%%%%%%%%%%%%%%%%%%%%%%%%%%%%%%%%%%%%%%%%%%%%%%%%%%%%%%%


\subsection{Beamer}
%%%%%%%%%%%%%%%%%%%%%%%%%%%%%%%%%%%%%%%%%%%%%%%%%%%%%%%%%%%%%%%%%%%%%%%%%%%%%%

\begin{frame}[fragile]
  \frametitle{Les blocs dans Beamer}

  \structure{Blocs libres}
  
  \begin{block}{Un bloc important}
    Les points suivants définissent l'importance

    \begin{itemize}
    \item un truc
    \item un bidule
    \end{itemize}
    
  \end{block}

  \pause  

  \structure{Définition, théorèmes et exemples}

  \begin{definition}
    Les cons ça osent tous c'est à ça qu'on les reconnait
  \end{definition}

  \begin{theorem}
    Il n'y a pas de plus grand nombre premier
  \end{theorem}

  \begin{example}
    Ce fichier est plein d'exemple à comparer au résultat en PDF
  \end{example}
\end{frame}

%%%%%%%%%%%%%%%%%%%%%%%%%%%%%%%%%%%%%%%%%%%%%%%%%%%%%%%%%%%%%%%%%%%%%%%%%%%%%%

\begin{frame}[fragile]
  \frametitle{D'autres fonctionnalités Beamer}

  \structure{Les recouvrements}
    
  \begin{itemize}
  \item \verb|\pause| pour pauser momentanément la diapo
    \pause
  \item<2> suffixer des commandes

    \begin{itemize}
    \item pour spécifier un truc apparaissant uniquement dans certains recouvrements

      \begin{itemize}
      \item $n$ via \verb|\item<|$n$\verb|>|
      \item $n$ à $m$ via \verb|\item<|$n-m$\verb|>|
      \item $n$ et $m$ via \verb|\item<|$n,m$\verb|>|
      \end{itemize}

    \item se place après certaines commandes (\verb|\item|, \verb|\alert|, etc.)
    \end{itemize}

  \end{itemize}
  
  \pause  
  
  \structure{Les mises en avant}
  
  \begin{itemize}
  \item On peut \alert{alerter} à propos de l'€
  \item On peut citer du texte
    
    \begin{quotation}
      Lorem ipsum dolor sit amet, consectetuer adipiscing elit. Donec hendrerit tempor tellus. Donec pretium posuere tellus. Proin quam nisl, tincidunt et, mattis eget, convallis nec, purus. Cum sociis natoque penatibus et magnis dis parturient montes, nascetur ridiculus mus. Nulla posuere. Donec vitae dolor. Nullam tristique diam non turpis. Cras placerat accumsan nulla. Nullam rutrum. Nam vestibulum accumsan nisl.
    \end{quotation}
    
  \end{itemize} 
\end{frame}

%%%%%%%%%%%%%%%%%%%%%%%%%%%%%%%%%%%%%%%%%%%%%%%%%%%%%%%%%%%%%%%%%%%%%%%%%%%%%%


\subsection{Références}
%%%%%%%%%%%%%%%%%%%%%%%%%%%%%%%%%%%%%%%%%%%%%%%%%%%%%%%%%%%%%%%%%%%%%%%%%%%%%%

\begin{frame}[fragile]
  \frametitle{Les outils à installer}


  \structure{Un système \TeX{} avec les paquets nécessaires}

  \begin{itemize}
  \item \TeX{} Live \hfill \url{http://www.tug.org/texlive}
  \item Mac\TeX{} \hfill \url{https://www.tug.org/mactex}
  \end{itemize}
  
  \pause

  \structure{Un éditeur facilitant le processus \emph{édition$\rightarrow$compilation$\rightarrow$visualisation}}

  \begin{itemize}
  \item Texmaker \hfill \url{http://www.xm1math.net/texmaker}
    
  \item Emacs avec AUC\TeX{} \hfill \url{https://www.gnu.org/software/emacs}
    
    \hfill \url{https://www.gnu.org/software/auctex}
    
  \end{itemize}
    
  \pause
  
  \structure{Un outil/service en ligne qui empaquete le tout (et un peu plus)}

  \begin{itemize}
  \item \textsc{ShareLaTeX} \hfill \url{https://www.sharelatex.com}
  \item $\backslash{}$BlueLaTeX \hfill \url{http://www.bluelatex.org}
  \end{itemize}
\end{frame}

%%%%%%%%%%%%%%%%%%%%%%%%%%%%%%%%%%%%%%%%%%%%%%%%%%%%%%%%%%%%%%%%%%%%%%%%%%%%%%

\begin{frame}[allowframebreaks]
  \frametitle<presentation>{Pour aller plus loin}
    
  \begin{thebibliography}{10}
    
    \beamertemplateonlinebibitems
    
  \bibitem{WikiBooks}
    \LaTeX{} Wikibooks
    \newblock \url{https://en.wikibooks.org/wiki/LaTeX}
      
  \bibitem{latexproject}
    The \LaTeX{} project
    \newblock \url{https://www.latex-project.org}

    \beamertemplatebookbibitems
    
  \bibitem{LaTeXBook}
    Leslie Lamport
    \newblock \emph{LaTeX: A document preparation system. User's guide and reference manual}
    \newblock Addison-Wesley, 1994
    
    \beamertemplatearticlebibitems

  \bibitem{Un papier}
    Prénom NOM
    \newblock Un article qui parle de LaTeX
    \newblock \emph{Journal of This and That}, 2(1):50--100,

  \end{thebibliography}
\end{frame}
  
%%%%%%%%%%%%%%%%%%%%%%%%%%%%%%%%%%%%%%%%%%%%%%%%%%%%%%%%%%%%%%%%%%%%%%%%%%%%%%

\againframe{titre} % Replace une diapo nommée avec l'option label

%%%%%%%%%%%%%%%%%%%%%%%%%%%%%%%%%%%%%%%%%%%%%%%%%%%%%%%%%%%%%%%%%%%%%%%%%%%%%%

\end{document}
